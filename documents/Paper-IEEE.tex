NGUYỄN THỊ HOÀNG KHÁNH, LÊ HOÀNG KHANG, TRẦN ĐÌNH KIÊN

\emph{Khoa Công nghệ thông tin, Trường Đại học Công nghiệp Thành phố Hồ
Chí Minh}

\href{mailto:nguyenthihoangkhanh@iuh.edu.vn}{\nolinkurl{nguyenthihoangkhanh@iuh.edu.vn}},

\href{mailto:hoangkhang.dev@gmail.com}{\nolinkurl{hoangkhang.dev@gmail.com}},
\href{mailto:tranndinhkien.dev@gmail.com}{\nolinkurl{tranndinhkien.dev@gmail.com}}

\textbf{Abstract.} Vietnam\textquotesingle s multi-specialty clinics
rely on outdated manual processes, including queuing for registration,
handwritten patient records prone to errors, manual clinical notes, and
cash/bank transfer payments. Staff manually cross-reference separate
ledgers, causing severe congestion during peak hours. These
inefficiencies prolong examinations, increase patient waiting times,
foster dissatisfaction, risk record mix-ups, and violate priority
principles for policy beneficiaries, the elderly, pregnant women, and
emergencies, delaying timely care.

With Vietnam\textquotesingle s population aging rapidly---projected to
surpass 20\% over 60 by 2030---coupled with rising chronic diseases and
modern lifestyle demands, the traditional system faces immense pressure.
Vietnam lags regionally in healthcare access, underscoring the need for
digital transformation per the national strategy by 2025.

Revita addresses these challenges as an intelligent management
application for multi-specialty clinics. It integrates AI to
auto-prioritize appointments per legal regulations, enables 24/7 online
booking, manages real-time doctor schedules with conflict alerts, and
streamlines registration, electronic records, and bank transfer
payments.

Additionally, Revita features an AI chatbot for direct patient
assistance during visits and an NLP-based diagnosis recommendation
system that extracts symptoms from notes to suggest accurate options.
This allows doctors to focus on treatment, reduces waiting times,
minimizes errors, and boosts efficiency, ensuring faster, equitable
healthcare for all.

\textbf{Keyword.} Medical Examination Process, Patient Record
Management, Medical Text-Based Disease Diagnosis, Next.js, Nest.js, AI
Chatbot, Recommendation System.

\textbf{Tóm tắt.} Các phòng khám đa khoa tại Việt Nam hiện vẫn phụ thuộc
vào quy trình thủ công lạc hậu, bao gồm xếp hàng đăng ký, ghi chép hồ sơ
bệnh nhân bằng tay dễ sai sót, viết tay phiếu khám bệnh và thanh toán
bằng tiền mặt/chuyển khoản. Nhân viên phải đối chiếu thủ công các sổ
sách riêng lẻ, dẫn đến tình trạng ùn tắc nghiêm trọng vào giờ cao điểm.
Những bất cập này làm kéo dài thời gian khám bệnh, tăng thời gian chờ
đợi của bệnh nhân, gây bất mãn, nguy cơ nhầm lẫn hồ sơ và vi phạm nguyên
tắc ưu tiên đối với người có công, người cao tuổi, phụ nữ mang thai và
trường hợp cấp cứu, làm chậm trễ việc chăm sóc kịp thời.

Với dân số Việt Nam đang già hóa nhanh chóng -- dự kiến vượt 20\% trên
60 tuổi vào năm 2030 -- cùng sự gia tăng bệnh mãn tính và nhu cầu lối
sống hiện đại, hệ thống truyền thống đang chịu áp lực cực lớn. Việt Nam
còn tụt hậu so với khu vực về tiếp cận dịch vụ y tế, do đó cần thúc đẩy
chuyển đổi số theo chiến lược quốc gia đến năm 2025.

Revita giải quyết các thách thức trên bằng ứng dụng quản lý thông minh
dành cho phòng khám đa khoa. Ứng dụng tích hợp AI để tự động ưu tiên
lịch hẹn theo quy định pháp luật, hỗ trợ đặt lịch trực tuyến 24/7, quản
lý lịch bác sĩ theo thời gian thực với cảnh báo xung đột, đồng thời đơn
giản hóa quy trình đăng ký, hồ sơ điện tử và thanh toán chuyển khoản.

Ngoài ra, Revita cung cấp chatbot AI hỗ trợ bệnh nhân trực tiếp trong
quá trình khám và hệ thống gợi ý chẩn đoán dựa trên NLP -- trích xuất
triệu chứng từ ghi chú để đề xuất phương án chính xác. Nhờ đó, bác sĩ
tập trung vào điều trị, giảm thời gian chờ, hạn chế sai sót, nâng cao
hiệu quả, đảm bảo dịch vụ y tế nhanh hơn, công bằng hơn cho mọi người.

\textbf{Từ khóa.} Quy trình khám bệnh, Quản lý hồ sơ bệnh nhân, Chẩn
đoán bệnh dựa trên văn bản y khoa, Next.js, Nest.js, Chatbot AI, Hệ
thống gợi ý.

\section{GIỚI THIỆU}\label{giux1edbi-thiux1ec7u}

Quy trình quản lý khám bệnh tại các phòng khám đa khoa ở Việt Nam bao
gồm nhiều giai đoạn tuần tự: lấy số thứ tự, hoàn tất thủ tục hành chính
để đăng ký và khám bệnh, chỉ định các dịch vụ chẩn đoán và thủ thuật cần
thiết, thanh toán chi phí, và cuối cùng là bác sĩ chẩn đoán tình trạng
bệnh, kê đơn thuốc cho bệnh nhân. Quy trình này đòi hỏi sự phối hợp chặt
chẽ giữa nhiều bên liên quan, bao gồm nhân viên hành chính, bác sĩ, bệnh
nhân và thu ngân. Với dân số già hóa nhanh chóng và nhu cầu y tế ngày
càng gia tăng, việc duy trì hiệu quả và công bằng trong quản lý trở nên
cực kỳ thách thức, đặc biệt khi các bước thủ công dẫn đến ùn tắc và sai
sót.

Để giải quyết các vấn đề này, chúng tôi đã phát triển và triển khai
\emph{\textbf{Revita: Hệ thống Quản lý Khám bệnh Thông minh Tích hợp Gợi
ý Chẩn đoán và Chatbot dành cho Phòng khám Đa khoa}} -- một giải pháp
công nghệ hiện đại nhằm tối ưu hóa và tự động hóa toàn bộ các giai đoạn
trong quy trình khám bệnh. Hệ thống không chỉ giảm tải công việc hành
chính mà còn nâng cao trải nghiệm tổng thể cho cả bệnh nhân và bác sĩ
trong suốt hành trình chăm sóc sức khỏe.

Revita được xây dựng trên các nền tảng web, di động và máy tính để bàn,
sử dụng các công nghệ tiên tiến như Next.js, Nest.js, React Native và
Electron. Điều này đảm bảo tính linh hoạt và khả năng truy cập, cho phép
người dùng tương tác mượt mà mọi lúc, mọi nơi -- từ đặt lịch khám trực
tuyến đến hiển thị số thứ tự theo thời gian thực trên màn hình kiosk.
Mục tiêu chính là nâng cao hiệu quả quản lý, cung cấp công cụ hỗ trợ tự
động và tạo ra môi trường hiện đại, thân thiện với bệnh nhân cũng như
nhân viên y tế.

Một tính năng nổi bật là hệ thống gợi ý chẩn đoán dựa trên ghi chú y
khoa, công cụ thông minh được thiết kế chuyên biệt để hỗ trợ bác sĩ phân
tích triệu chứng và đề xuất các tình trạng bệnh phù hợp. Hệ thống sử
dụng xử lý ngôn ngữ tự nhiên (NLP) để trích xuất thông tin chính từ văn
bản lâm sàng, kết hợp với thuật toán gợi ý dựa trên nội dung nhằm đưa ra
các đề xuất chính xác và nhanh chóng -- từ đó nâng cao độ tin cậy trong
chẩn đoán.

\section{CƠ SỞ LÝ THUYẾT}\label{cux1a1-sux1edf-luxfd-thuyux1ebft}

\subsection{Kiến trúc phần
mềm}\label{kiux1ebfn-truxfac-phux1ea7n-mux1ec1m}

\begin{figure}
\centering
\includegraphics[width=3.00678in,height=2.76316in]{media/image1.png}
\caption{Hình 1 Sơ đồ kiến trúc của phần mềm}
\end{figure}

\textbf{Tầng giao diện người dùng (Client Layer):} bao gồm ba ứng dụng
độc lập về nền tảng nhưng thống nhất về trải nghiệm người dùng. Ứng dụng
web được xây dựng trên Next.js với kiến trúc App Router hiện đại; ứng
dụng di động được phát triển bằng React Native nhằm tối ưu hiệu năng
trên cả hai hệ điều hành chính; ứng dụng desktop được đóng gói bằng
Electron để đáp ứng nhu cầu xử lý tác vụ nặng tại các phòng khám. Các
ứng dụng này chỉ chịu trách nhiệm hiển thị dữ liệu và thu thập tương tác
người dùng, không chứa bất kỳ logic nghiệp vụ nào.

\textbf{Tầng trình diễn (Presentation Layer)}: được thực hiện bởi Nginx
đóng vai trò cổng giao tiếp duy nhất với thế giới bên ngoài. Nginx đảm
nhiệm các chức năng chấm dứt kết nối bảo mật, định tuyến yêu cầu theo
kênh người dùng, cân bằng tải, giới hạn tần suất truy cập và chuyển tiếp
giao thức WebSocket mà không làm gián đoạn luồng dữ liệu thời gian thực.

\textbf{Tầng ứng dụng (Application Layer):} bao gồm các instance NestJS
được tổ chức theo mô hình Backend-for-Frontend. Mỗi instance phục vụ
riêng một kênh người dùng thông qua tập hợp các Controller chuyên biệt,
thực hiện việc tiếp nhận yêu cầu, xác thực phiên làm việc, xác thực dữ
liệu đầu vào và điều phối luồng xử lý xuống các tầng bên dưới. Việc tách
biệt này cho phép tối ưu hóa cấu trúc phản hồi theo nhu cầu riêng của
từng nền tảng mà không làm phức tạp hóa giao diện lập trình ứng dụng.

\textbf{Tầng dịch vụ (Service Layer)}: tập trung toàn bộ quy tắc nghiệp
vụ cốt lõi của hệ thống quản lý bệnh viện. Các Service tại tầng này được
triển khai dưới dạng các đơn vị xử lý thuần túy, không phụ thuộc vào cơ
sở dữ liệu hay giao thức truyền tải. Tại đây diễn ra các quyết định
nghiệp vụ quan trọng như kiểm tra ràng buộc lịch khám, tính toán quyền
truy cập hồ sơ, sinh đơn thuốc điện tử và xử lý quy trình thanh toán,
đảm bảo tính nhất quán và khả năng kiểm thử độc lập.

\textbf{Tầng truy cập dữ liệu (Data Access Layer):} được xây dựng trên
Prisma Client kết hợp với Repository Pattern. Tầng này đóng vai trò
trung gian duy nhất giữa logic nghiệp vụ và các hệ cơ sở dữ liệu, cung
cấp giao diện thống nhất và an toàn kiểu dữ liệu cho mọi thao tác đọc
ghi. Nhờ cơ chế sinh mã tự động, mọi thay đổi trong mô hình dữ liệu đều
được phản ánh tức thì trên toàn hệ thống mà không yêu cầu chỉnh sửa thủ
công.

\textbf{Tầng lưu trữ (Persistence Layer)}: sử dụng đồng thời ba hệ quản
trị cơ sở dữ liệu phù hợp với đặc thù dữ liệu. PostgreSQL đảm bảo tính
toàn vẹn và nhất quán cho các thực thể có quan hệ phức tạp; MongoDB phục
vụ lưu trữ tài liệu phi cấu trúc và nhật ký hoạt động với hiệu năng ghi
cao; Redis được sử dụng làm bộ nhớ đệm và cơ chế lưu trữ phiên thời gian
thực, hỗ trợ các tính năng thông báo tức thì và trò chuyện trực tuyến.

\textbf{Tầng hạ tầng (Infrastructure Layer):} bao gồm tập hợp các dịch
vụ bên thứ ba mà hệ thống tích hợp. Các dịch vụ này bao gồm nền tảng lưu
trữ đối tượng trên Supabase Storage và DigitalOcean Spaces, cổng thanh
toán PayOS và hệ thống email giao dịch Resend. Tất cả đều được đóng gói
dưới dạng các Provider có giao diện thống nhất, cho phép thay đổi nhà
cung cấp mà không ảnh hưởng đến các tầng nghiệp vụ phía trên.

\subsection{Hệ thống chẩn đoán
bệnh}\label{hux1ec7-thux1ed1ng-chux1ea9n-ux111ouxe1n-bux1ec7nh}

\subsubsection{TF-IDF}\label{tf-idf}

Trọng số TF-IDF mô hình hoá mức độ ``thông tin'' của n-gram bằng cách
kết hợp cường độ xuất hiện tại tài liệu với độ hiếm trên toàn bộ tập. Để
giảm thiên lệch theo độ dài và ổn định số, dùng TF dưới tuyến tính và
chuẩn hoá \(L_{2}\)sau cùng. Cho thuật ngữ \(t\ \)trong tài liệu \(d\),
với \(N\) là số tài liệu huấn luyện và \(df(t)\)là số tài liệu chứa
\(t\):

\[{idf(t) = \log(\frac{1 + N}{1 + df(t)}) + 1
}{tfidf(t,d) = {tf}'(t,d) \cdot idf(t)
}{{\widehat{\mathbf{x}}}_{d} = \frac{\mathbf{x}_{d}}{\parallel \mathbf{x}_{d} \parallel_{2}}}\]

Trong thiết lập dự đoán ICD từ discharge note, biểu diễn TF-IDF thưa
(vocabulary lớn) được ghép với thuộc tính tuổi (chuẩn hoá) và giới tính
(one-hot) để tạo biểu diễn đầu vào thống nhất cho các bộ phân loại nhị
phân ở các bước sau.

\subsubsection{One-vs-Rest (OvR)
reduction}\label{one-vs-rest-ovr-reduction}

Chiến lược One-Vs-Rest (OvR) ánh xạ bài toán đa nhãn thành tập các bài
toán nhị phân độc lập, mỗi nhãn \(k\)đi kèm một bộ phân loại tuyến tính
riêng. Với đầu vào \(\mathbf{x} \in \mathbb{R}^{F}\), điểm tuyến tính
\(z_{k} = \mathbf{w}_{k}^{\top}\mathbf{x} + b_{k}\)được chuyển thành xác
suất hậu nghiệm \(p_{k} = \sigma(z_{k}) = 1/(1 + e^{- z_{k}})\). Cấu
trúc này cho phép huấn luyện song song theo nhãn và suy luận đồng thời
bằng phép nhân ma trận
\(\mathbf{p} = \sigma(\mathbf{Wx} + \mathbf{b})\), trong đó
\(\mathbf{W} \in \mathbb{R}^{K \times F}\)và
\(\mathbf{b} \in \mathbb{R}^{K}\). Việc chọn nhãn đầu ra có thể dựa trên
ngưỡng
\(\tau\)(\({\widehat{y}}_{k} = \mathbb{I}\lbrack p_{k} \geq \tau\rbrack\))
hoặc Top-\(K'\) xác suất lớn nhất. Đối với bộ mã ICD quy mô lớn, OvR duy
trì tính tuyến tính theo số nhãn, đồng thời giữ khả năng diễn giải: hệ
số lớn theo chiều của một n-gram phản ánh tín hiệu mạnh cho nhãn tương
ứng {[}3{]}.

\subsubsection{SGD (Stochastic Gradient Descent)
optimizer}\label{sgd-stochastic-gradient-descent-optimizer}

Tham số của từng bộ phân loại OvR được ước lượng bằng hạ gradient ngẫu
nhiên với mất mát logistic và chuẩn hoá \(L_{2}\), phù hợp dữ liệu thưa
và khối lượng lớn. Với mẫu \((\mathbf{x}_{i},y_{i,k})\), xác suất

\[{p_{i,k} = \sigma\left( \mathbf{w}_{\mathbf{k}}^{\top\mathbf{x}_{\mathbf{i}}} + b_{k} \right)\ 
}{\sigma(z) = \frac{1}{1 + e^{- z}}.}\]

và hàm mục tiêu

\[\mathcal{L}_{\mathcal{k}}\left( \mathbf{w}_{\mathbf{k}},b_{k} \right) = \frac{1}{n}\sum_{i = 1}^{n}\left\lbrack - y_{i,k}\log p_{i,k} - \left( 1 - y_{i,k} \right)\log\left( 1 - p_{i,k} \right) \right\rbrack + \frac{\lambda}{2}\text{|}\mathbf{w}_{\mathbf{k}}\text{|}_{2}^{2}.\]

trong đó \(\lambda > 0\)kiểm soát mức phạt để hạn chế quá khớp. Với
minibatch \(B\), gradient được xấp xỉ

\[g_{w_{k}} = \frac{1}{|B|}\sum_{i \in B}^{}\left( p_{i,k} - y_{i,k} \right)\, x_{i} + \lambda w_{k},\quad\]

\[g_{b_{k}} = \frac{1}{|B|}\sum_{i \in B}^{}\left( p_{i,k} - y_{i,k} \right).
\]

và cập nhật

\[\mathbf{w}_{k} \leftarrow \mathbf{w}_{k} - \alpha\text{ }g_{\mathbf{w}_{k}},b_{k} \leftarrow b_{k} - \alpha\text{ }g_{b_{k}},
\]

với tốc độ học \(\alpha\). Khi áp dụng cho discharge note, biểu diễn
TF-IDF kết hợp thuộc tính tuổi/giới tính tạo điều kiện hội tụ ổn định
của SGD, trong khi ma trận trọng số \(\mathbf{W}\) cho phép suy luận
hàng loạt và giải thích tuyến tính mức đặc trưng {[}4{]}.

\subsection{Công nghệ sử dụng trong hệ
thống}\label{cuxf4ng-nghux1ec7-sux1eed-dux1ee5ng-trong-hux1ec7-thux1ed1ng}

\begin{itemize}
\item
  \textbf{NextJS} được sử dụng để xây dựng giao diện web quản trị và đặt
  lịch khám trực tuyến 24/7, tận dụng Server-Side Rendering (SSR) và
  Static Site Generation (SSG) nhằm tối ưu tốc độ tải trang, cải thiện
  SEO và đảm bảo trải nghiệm người dùng mượt mà trên mọi thiết bị. Tích
  hợp API Routes giúp xử lý backend ngay trong frontend, giảm độ trễ và
  đơn giản hóa triển khai.
\item
  \textbf{NestJS} xây dựng hệ thống backend trung tâm của Revita, quản
  lý toàn bộ logic nghiệp vụ như lịch khám, hồ sơ bệnh nhân, ưu tiên
  theo quy định pháp luật và thanh toán chuyển khoản. Kiến trúc modular
  kết hợp Dependency Injection giúp dễ mở rộng, bảo trì; hỗ trợ
  REST/GraphQL và kết nối an toàn với cơ sở dữ liệu PostgreSQL hoặc
  MongoDB.
\item
  \textbf{React Native} phát triển ứng dụng di động cho bệnh nhân và bác
  sĩ, cho phép đặt lịch, xem số thứ tự, nhận thông báo, tra cứu hồ sơ
  trên cả iOS và Android từ một mã nguồn duy nhất. Tận dụng hot
  reloading để cập nhật nhanh, truy cập native API như camera, GPS để hỗ
  trợ chụp ảnh bệnh án hoặc định vị phòng khám.
\item
  \textbf{Electron} xây dựng ứng dụng desktop cho quầy lễ tân và kiosk
  tự phục vụ, hiển thị số thứ tự theo thời gian thực, in phiếu khám,
  quản lý thanh toán tại chỗ. Chạy độc lập trên Windows, macOS, Linux,
  tích hợp trực tiếp với máy in, máy quét mã QR và hệ thống loa phát
  thanh mà không cần trình duyệt.
\item
  \textbf{FastAPI} triển khai microservices AI, xử lý gợi ý chẩn đoán từ
  ghi chú y khoa và chatbot hỗ trợ bệnh nhân. Tận dụng async kết hợp
  Pydantic để xử lý hàng nghìn yêu cầu đồng thời với độ trễ thấp, tự
  động sinh OpenAPI docs để tích hợp nhanh với frontend, phù hợp cho mô
  hình NLP và ML trong môi trường sản xuất.
\item
  \textbf{Gemini 2.5 Flash} tích hợp làm lõi AI thông minh cho chatbot
  tư vấn 24/7 và hệ thống gợi ý chẩn đoán đa phương thức. Phân tích văn
  bản, hình ảnh bệnh án, âm thanh mô tả triệu chứng trong context 1M
  token, đưa ra gợi ý chính xác, nhanh chóng; hỗ trợ suy luận nội bộ để
  giải thích kết quả, đảm bảo độ tin cậy cao trong môi trường y tế thực
  tế.
\item
  \textbf{PostgreSQL} được sử dụng làm cơ sở dữ liệu chính để quản lý
  toàn bộ hệ thống nghiệp vụ lớn của Revita, bao gồm hồ sơ bệnh nhân,
  lịch khám, thông tin bác sĩ, thanh toán và ưu tiên theo quy định pháp
  luật. Với khả năng xử lý giao dịch phức tạp, hỗ trợ ACID và mở rộng
  theo chiều dọc, PostgreSQL đảm bảo tính toàn vẹn dữ liệu và hiệu suất
  ổn định trong môi trường phòng khám hoạt động liên tục với lưu lượng
  cao.
\item
  \textbf{Redis} được triển khai như bộ nhớ đệm (cache) hiệu năng cao và
  sử dụng Redis Streams để lưu trữ luồng sự kiện theo thời gian thực như
  bốc số thứ tự, cập nhật hàng chờ, thông báo trạng thái khám. Nhờ tốc
  độ truy xuất dưới mili giây và cơ chế pub/sub, Redis giúp hiển thị số
  thứ tự tức thì trên kiosk, ứng dụng di động và quầy lễ tân, giảm tải
  cho database chính và đảm bảo trải nghiệm mượt mà trong giờ cao điểm.
\item
  \textbf{MongoDB} được dùng để lưu trữ và quản lý dữ liệu thuốc với
  khối lượng lớn, cấu trúc linh hoạt (tên thuốc, hoạt chất, liều lượng,
  chỉ định, tương tác). Nhờ cơ chế lập chỉ mục mạnh mẽ và tìm kiếm
  full-text, MongoDB tối ưu hóa tốc độ tra cứu thuốc theo nhiều tiêu
  chí, hỗ trợ bác sĩ kê đơn nhanh chóng, chính xác và giảm thiểu sai sót
  trong quy trình khám bệnh.
\end{itemize}

\subsection{Dataset MIMIC-IV}\label{dataset-mimic-iv}

MIMIC-IV là một bộ dữ liệu hồ sơ y tế điện tử (EHR) mã nguồn mở, được
phát triển bởi nhóm nghiên cứu tại Beth Israel Deaconess Medical Center
(BIDMC) và MIT Laboratory for Computational Physiology vào năm 2020, như
một bản cập nhật từ phiên bản MIMIC-III. Bộ dữ liệu này áp dụng kiến
trúc modular với các module riêng biệt như hosp (dữ liệu bệnh viện), icu
(dữ liệu đơn vị chăm sóc tích cực), và ed (dữ liệu cấp cứu), được tổ
chức bằng các bảng SQL để nhấn mạnh nguồn gốc dữ liệu và hỗ trợ tích hợp
linh hoạt giữa các nguồn khác nhau. Dữ liệu được thu thập từ hệ thống
EHR tùy chỉnh, hệ thống thông tin ICU (MetaVision), và các nguồn bên
ngoài, sau đó được biến đổi và khử nhận dạng theo quy định HIPAA. Chức
năng chính của MIMIC-IV là cung cấp hơn 250.000 hồ sơ bệnh nhân từ năm
2008-2019, bao gồm đo lường sinh lý, đơn thuốc, chẩn đoán, thủ thuật, và
ghi chú lâm sàng, nhằm hỗ trợ nghiên cứu trong lĩnh vực thông tin lâm
sàng, dịch tễ học, và học máy, đặc biệt trong phân tích chăm sóc bệnh
nhân cấp cứu và ICU {[}1{]}.

\begin{figure}
\centering
\includegraphics[width=2.75486in,height=1.79999in]{media/image2.png}
\caption{Biểu đồ 1 - Phân bố độ tuổi nhập viện}
\end{figure}

Phân bố tuổi bệnh nhân trong MIMIC-IV có dạng đa đỉnh (multimodal), tập
trung mạnh ở nhóm 50--80 tuổi, với đỉnh cao nhất tại \textasciitilde65
tuổi -- phản ánh dân số bệnh nhân ICU chủ yếu là người trung niên và cao
tuổi.

\begin{figure}
\centering
\includegraphics[width=2.75477in,height=1.79893in]{media/image3.png}
\caption{Biểu đồ 2 - Phân bố độ dài của các bệnh án y khoa}
\end{figure}

\includegraphics[width=2.625in,height=1.79868in]{media/image4.png}

Trong MIMIC-IV, chỉ 100 mã ICD phổ biến nhất đã chiếm
\textasciitilde65\% tổng số chẩn đoán, và 1.000 mã chiếm
\textasciitilde90\%. Đường cong tăng nhanh ở đầu và chậm dần về sau thể
hiện hiện tượng đuôi dài mạnh. Với Revita, việc tập trung huấn luyện
trên top 200--500 mã đảm bảo độ chính xác cao cho phần lớn ca bệnh thực
tế, đồng thời tối ưu tài nguyên khi triển khai hệ thống gợi ý chẩn đoán.

\begin{figure}
\centering
\includegraphics[width=2.39246in,height=1.91608in]{media/image5.png}
\caption{Biểu đồ 3 - Tỉ lệ giới tính của bệnh nhân}
\end{figure}

Trong MIMIC-IV, nữ giới chiếm 52.7\% (191.383 bệnh nhân) và nam giới
chiếm 47.3\% (172.643 bệnh nhân) trong tổng số 364.026 bệnh nhân duy
nhất. Tỷ lệ này cho thấy sự cân bằng tương đối giữa hai giới, với nữ
giới nhỉnh hơn nhẹ -- phù hợp với đặc điểm dân số nhập viện ICU. Với
Revita, hệ thống gợi ý chẩn đoán và quản lý hồ sơ được thiết kế không
thiên vị giới tính, đảm bảo độ chính xác đồng đều trên cả hai nhóm bệnh
nhân.

\section{\texorpdfstring{Hệ thống quản lý quy trình khám chữa bệnh
}{Hệ thống quản lý quy trình khám chữa bệnh }}\label{hux1ec7-thux1ed1ng-quux1ea3n-luxfd-quy-truxecnh-khuxe1m-chux1eefa-bux1ec7nh}

\subsection{Quy trình khám chữa bệnh tại phòng
khám}\label{quy-truxecnh-khuxe1m-chux1eefa-bux1ec7nh-tux1ea1i-phuxf2ng-khuxe1m}

\textbf{Bước 1}: Bệnh nhân có thể đặt lịch khám trước sau đó đến phòng
khám, nhập thông tin khám bệnh nếu đăng ký khám lần đầu, nếu đã từng đến
khám tại bệnh viện thì bệnh nhân nhập mã hồ sơ khám/ số điện thoại hồ
sơ/mã đặt lịch (nếu có) sau đó hệ thống sẽ phân phối bệnh nhân vào các
quầy làm thủ tục khám với cơ chế xoay vòng có ưu tiên.

\textbf{Bước 2}: Nhân viên quầy thủ tục nhấn gọi bệnh nhân đến quầy làm
thủ tục, bệnh nhân quan sát thông tin của mình trên màn hình chờ để biết
khi nào chuẩn bị đến làm thủ tục.

\textbf{Bước 3}: Bệnh nhân sau khi làm thủ tục khám thì cầm theo phiếu
chỉ định các dịch vụ đến quầy thu ngân nộp để thanh toán các phí dịch vụ
theo yêu cầu. Nhân viên thu nhân nhận phiếu chỉ định và bắt đầu tạo hóa
đơn thanh toán với các dịch vụ bệnh nhân muốn thực hiện . Có hai phương
thức thanh toán là bằng tiền mặt (nhân viên thu ngân xác nhận thanh toán
thành công khi nhận tiền) và thanh toán qua hình thức chuyển khoản (hệ
thống tự động cập nhật thông tin và xuất hóa đơn khi bệnh nhân chuyển
khoản thành công).

\textbf{Bước 4}: Nếu bệnh nhân đặt lịch trước đó hoặc là yêu cầu chọn
bác sĩ khi ở quầy thủ tục thì khi thanh toán thành công bệnh nhân sẽ
được thêm vào hàng chờ của bác sĩ đó. Nếu không thì bệnh nhân sẽ đến
trước khu vực khám và checkin tại các máy kiosk hoặc nhờ nhân viên ở tại
quầy thủ tục (khu vực khám bệnh) để xác nhận vào khám, khi xác nhận xong
thì hệ thống đưa bệnh nhân vào hàng chờ.

\textbf{Bước 5}: Bác sĩ nhấn gọi bệnh nhân vào phòng khám theo thứ tự
hàng chờ, sau khi vào khám thì bác sĩ tiến hành chẩn đoán, tạo bệnh án,
đơn thuốc cho bệnh nhân. Nếu bác sĩ mong muốn bệnh nhân thực hiện thêm
các dịch vụ cho mục đích khám chữa bệnh thì bác sĩ sẽ tạo phiếu chỉ định
cho bệnh nhân thực hiện các dịch vụ đó, sau đó quay về bước 3.

\textbf{Bước 6}: Khi đã hoàn thành các dịch vụ theo yêu cầu của bác sĩ
thì bệnh nhân tiến hành quay lại khu vực khám để checkin chờ kết quả.

\includegraphics[width=3.41797in,height=4.53754in]{media/image6.png}

\subsection{Một số chức năng
chính}\label{mux1ed9t-sux1ed1-chux1ee9c-nux103ng-chuxednh}

Một số chức năng chính của hệ thống:

\begin{itemize}
\item
  Đề xuất/Xác nhận lịch làm việc.
\item
  Đặt lịch hẹn khám bệnh
\item
  Lấy số thứ tự đến quầy thủ tục
\item
  Quản lý phiếu chỉ định, bệnh án, đơn thuốc
\item
  Hệ thống chatbot hỗ trợ bệnh nhân (Không hỗ trợ chẩn đoán, đơn thuốc,
  \ldots{} các yếu tố có thể ảnh hưởng đến sức khỏe của bệnh nhân.
\item
  Hệ thống gợi ý các chẩn đoán dựa trên bệnh án của bệnh nhân (chỉ cho
  các bác sĩ trong hệ thống sử dụng)
\end{itemize}

\includegraphics[width=3.49931in,height=3.08383in]{media/image7.png}

\subsection{Sơ đồ triển khai hệ
thống}\label{sux1a1-ux111ux1ed3-triux1ec3n-khai-hux1ec7-thux1ed1ng}

\includegraphics[width=3.32632in,height=3.97639in]{media/image8.png}

\section{TRIỂN KHAI}\label{triux1ec3n-khai}

\subsection{Một số màn hình giao
diện}\label{mux1ed9t-sux1ed1-muxe0n-huxecnh-giao-diux1ec7n}

\begin{figure}
\centering
\includegraphics[width=2.74544in,height=1.77804in]{media/image9.png}
\caption{Hình 2 Chuẩn đoán tự động dựa trên thông tin đề xuất}
\end{figure}

\begin{figure}
\centering
\includegraphics[width=2.66667in,height=1.60114in]{media/image10.png}
\caption{Hình 3 Tra cứu thông tin khám bệnh bằng chatbot AI}
\end{figure}

\subsection{Kết quả thực
nghiệm}\label{kux1ebft-quux1ea3-thux1ef1c-nghiux1ec7m}

Mô hình được huấn luyện theo hướng phân loại đa nhãn (multi-label) nhằm
dự đoán các mã ICD bệnh lý dựa trên tóm tắt ra viện (discharge summary).
Phương pháp: One-vs-Rest Logistic Regression.

\begin{itemize}
\item
  Đặc trưng đầu vào: TF-IDF n-gram (1--2 từ) trên văn bản đã làm sạch.
\item
  Dữ liệu huấn luyện: \textasciitilde200 k hồ sơ MIMIC-IV.
\item
  Đầu ra: xác suất của từng mã ICD (≈ 2k nhãn)
\item
  Chiến lược đánh giá: top-K (1, 3, 5, 10)
\item
  Bảng 4.1 Kết quả định lượng
\end{itemize}

{\def\LTcaptype{none} % do not increment counter
\begin{longtable}[]{@{}
  >{\raggedright\arraybackslash}p{(\linewidth - 6\tabcolsep) * \real{0.1504}}
  >{\raggedright\arraybackslash}p{(\linewidth - 6\tabcolsep) * \real{0.0964}}
  >{\raggedright\arraybackslash}p{(\linewidth - 6\tabcolsep) * \real{0.0137}}
  >{\raggedright\arraybackslash}p{(\linewidth - 6\tabcolsep) * \real{0.2259}}@{}}
\toprule\noalign{}
\begin{minipage}[b]{\linewidth}\raggedright
\textbf{Chỉ số}
\end{minipage} &
\multicolumn{2}{>{\centering\arraybackslash}p{(\linewidth - 6\tabcolsep) * \real{0.1102} + 2\tabcolsep}}{%
\begin{minipage}[b]{\linewidth}\centering
\textbf{Kết quả}
\end{minipage}} & \begin{minipage}[b]{\linewidth}\centering
\textbf{Giải thích}
\end{minipage} \\
\midrule\noalign{}
\endhead
\bottomrule\noalign{}
\endlastfoot
Tổng số ca kiểm tra & 50 &
\multicolumn{2}{>{\raggedright\arraybackslash}p{(\linewidth - 6\tabcolsep) * \real{0.2396} + 2\tabcolsep}@{}}{%
Mẫu từ tập test} \\
Hit@1 & 76\% &
\multicolumn{2}{>{\raggedright\arraybackslash}p{(\linewidth - 6\tabcolsep) * \real{0.2396} + 2\tabcolsep}@{}}{%
76 \% ca có ít nhất 1 mã đúng ở vị trí đầu tiên} \\
Hit@3 & 94\% &
\multicolumn{2}{>{\raggedright\arraybackslash}p{(\linewidth - 6\tabcolsep) * \real{0.2396} + 2\tabcolsep}@{}}{%
94 \% ca có ít nhất 1 mã đúng trong 3 mã đầu} \\
Hit@5 & 96\% &
\multicolumn{2}{>{\raggedright\arraybackslash}p{(\linewidth - 6\tabcolsep) * \real{0.2396} + 2\tabcolsep}@{}}{%
96 \% ca có ít nhất 1 mã đúng trong top-5} \\
Hit@10 & 98\% &
\multicolumn{2}{>{\raggedright\arraybackslash}p{(\linewidth - 6\tabcolsep) * \real{0.2396} + 2\tabcolsep}@{}}{%
49/50 ca có ít nhất 1 mã đúng trong top-10} \\
Trung bình số mã đúng / ca (avg hits @ 10) & 2.62 &
\multicolumn{2}{>{\raggedright\arraybackslash}p{(\linewidth - 6\tabcolsep) * \real{0.2396} + 2\tabcolsep}@{}}{%
Mỗi ca dự đoán trúng trung bình \textasciitilde2.6 bệnh} \\
Số ca dự đoán đúng hoàn toàn & 22/50 (44\%) &
\multicolumn{2}{>{\raggedright\arraybackslash}p{(\linewidth - 6\tabcolsep) * \real{0.2396} + 2\tabcolsep}@{}}{%
Toàn bộ mã thật đều nằm trong top-10} \\
\end{longtable}
}

\section{KẾT LUẬN}\label{kux1ebft-luux1eadn}

Trong đồ án này, chúng tôi đã trình bày chi tiết quy trình xây dựng và
triển khai hệ thống Revita -- một nền tảng hỗ trợ chẩn đoán y khoa trên
cả web và di động. Hệ thống tích hợp các công nghệ hiện đại như NextJS,
NestJS, Electron, React Native, FastAPI, Gemini, kết hợp với mô hình dự
đoán bệnh được huấn luyện từ tập dữ liệu y khoa nổi tiếng MIMIC-IV. Nhờ
đó, hệ thống cung cấp gợi ý bệnh lý chính xác, nhanh chóng dựa trên ghi
chú bệnh án, tuổi và giới tính, hỗ trợ hiệu quả cho bác sĩ trong công
tác sàng lọc và chẩn đoán ban đầu.

Mặc dù đã đạt được những kết quả khả quan, hệ thống vẫn còn hạn chế lớn
là chưa sử dụng dữ liệu thực tế từ chính người dùng để huấn luyện và cải
thiện mô hình. Trong thời gian tới, nhóm sẽ tập trung:

Thu thập và tích hợp dữ liệu nội sinh từ hệ thống Revita để tái huấn
luyện mô hình định kỳ, nâng cao độ chính xác và phù hợp với thực tế lâm
sàng tại Việt Nam.

Cải thiện trải nghiệm người dùng: tối ưu giao diện, tăng tốc độ phản
hồi, hỗ trợ đa ngôn ngữ và tích hợp voice-to-text cho ghi chú bệnh án.

Nâng cấp mô hình dự đoán: thử nghiệm các mô hình ngôn ngữ y khoa chuyên
biệt (như BioClinicalBERT), bổ sung xử lý phủ định và ngữ cảnh phức tạp.

Mở rộng nền tảng: tích hợp với hệ thống bệnh viện (HIS), và triển khai
thử nghiệm thực tế tại các cơ sở y tế.

TÀI LIỆU THAM KHẢO

{[}1{]} A. E. Johnson \emph{et al.}, ``MIMIC-IV, a freely accessible
electronic health record dataset,'' \emph{Sci. Data}, vol. 10, no. 1, p.
1, 2023.

{[}2{]} C. D. Manning, P. Raghavan, and H. Schütze, ``Scoring, term
weighting and the vector space model (Chap. 6),'' in \emph{Introduction
to information retrieval}. Cambridge, U.K.: Cambridge Univ. Press, 2008,
pp. 110--134.

{[}3{]} V. Ashwinkumar, P. P. Arage, R. Jeya, and P. Sudhakaran,
``One-vs-Rest vs. Voting Classifiers for Multi-Label Text
Classification: An Empirical Study,'' in \emph{E3S Web of Conferences},
vol. 491, p. 01014, 2024.

{[}4{]} H. Robbins and S. Monro, ``A stochastic approximation method,''
\emph{Ann. Math. Statist.}, pp. 400--407, 1951.

\section{}\label{section}
